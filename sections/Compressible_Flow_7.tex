\section{Prandtl关系}

再次考虑一维管道内的超音速流动问题。

\begin{align*}
    \rho_{1} u_{1}&=\rho_{2} u_{2} &\mbox{连续性方程}\\ 
    p_{1}+\rho_{1} u_{1}^{2}&=p_{2}+\rho_{1} u_{2}^{2}&\mbox{动量方程}\\ 
    \frac{u_{1}^{2}}{2}+\frac{\gamma}{\gamma-1} \frac{p_{1}}{\rho_{1}}&=\frac{u_{2}^{2}}{2}+\frac{\gamma}{\gamma-1} \frac{p_{2}}{\rho_{2}}&\mbox{能量方程}
\end{align*}

给定$M_1(u_1)\ p_1\ \rho_1$,通过三个方程求解三个未知数$M_2(u_2)\ p_2\ \rho_2$。

\subsection{正激波推导的另一种方法}

动量式除以连续性式

\begin{align*}
    \frac{p_{1}}{\rho_{1} u_{1}}+u_{1}&=\frac{p_{2}}{\rho_{2} u_{2}}+u_{2}\\ 
    u_{1}-u_{2}&=\frac{a_{2}^{2}}{\gamma u_{2}}-\frac{a_{1}^{2}}{\gamma u_{1}}
\end{align*}

由能量式

\begin{equation*}
    \frac{u^{2}}{2}+\frac{a^{2}}{\gamma-1}=K\mbox{(常数)}
\end{equation*}

设喉部达到声速$a^{*}$,则有

\begin{align*}
    K&=\frac{a^{* 2}}{2}+\frac{a^{*}}{\gamma-1}\\ 
    &=\frac{1}{2} \frac{(\gamma+1)}{(\gamma-1)} a^{* 2}
\end{align*}

\begin{align*}
\begin{array}{l}{a_{1}^{2}=\frac{1}{2}\left\{(\gamma+1) a^{* 2}-(\gamma-1) u_{1}^{2}\right\}} \\ {a_{2}^{2}=\frac{1}{2}\left\{(\gamma+1) a^{* 2}-(\gamma-1) u_{2}^{2}\right\}}\end{array}
\end{align*}

代回至速度差式,则有

\begin{gather*}
    u_{1}-u_{2}=\frac{(\gamma+1) a^{* 2}-(\gamma-1) u_{2}^{2}}{2 \gamma u_{2}}-\frac{(\gamma+1) a^{* 2}-(\gamma-1) u_{1}^{2}}{2 \gamma u_{1}}
\end{gather*}

化简可以得到

\begin{equation*}
    u_{1} u_{2}=a^{* 2}
\end{equation*}

定义$M_{1}^{*}=u_{1} / a^{*}$,$M_{2}^{*}=u_{2} / a^{*}$,可以得到

\begin{equation*}
    M_{1}^{*}=\frac{1}{M_{2}^{*}}
\end{equation*}

这就是Prandtl-Meyer关系。

\subsection{$M$与$M^*$的关系}

利用能量关系除以$u^2$,

\begin{gather*}
    M^{* 2}=\frac{(\gamma+1) M^{2}}{(\gamma-1) M^{2}+2}\\ 
    M_{2}^{2}=\frac{\frac{\gamma-1}{2} M_{1}^{2}+1}{\gamma M_{1}^{2}-\frac{\gamma-1}{2}}
\end{gather*}

连续性方程

\begin{align*}
    \frac{\rho_2}{\rho_1}&=\frac{u_1}{u_2}=\frac{u_1^2}{u_1u_2}=\frac{M_1^{*2}a^{*2}}{M_1^*M_2^*a^{*2}}=M_1^{*2}\\ 
    \frac{\rho_{2}}{\rho_{1}}&=\frac{(\gamma+1) M_{1}^{2}}{(\gamma-1) M_{1}^{2}+2}
\end{align*}

动量方程

\begin{align*}
    p_{2}-p_{1}&=\rho_{1} u_{1}\left(u_{1}-u_{2}\right)\\ 
    \frac{p_{2}-p_{1}}{p_{1}}&=\gamma M_{1}^{2}\left(1-\frac{u_{2}}{u_{1}}\right)\\ 
    \frac{p_{2}}{p_{1}}-1&=\frac{2 \gamma}{\gamma+1}\left(M_{1}^{2}-1\right)
\end{align*}

求解温度则可以利用

\begin{equation*}
    \frac{T_{2}}{T_{1}}=\frac{p_{2}}{p_{1}} \cdot \frac{\rho_{1}}{\rho_{2}}
\end{equation*}