\section{等熵可压缩流}

\subsection{可压缩流体的欧拉方程}

\begin{gather*}
    \frac{a^{2}}{\gamma-1}+\frac{V^{2}}{2}=\mathrm{const}\\
    \frac{p}{\rho} \frac{\gamma}{\gamma-1}+\frac{1}{2} V^{2}=\text { const }\\ 
    \frac{1}{2} V^{2}+T c_{p}=\text { const }
\end{gather*}

滞止温度与滞止焓:

\begin{align*}
    \frac{1}{2} V^{2}+T c_{p}&=T_{0} c_{p}\\ 
    \frac{1}{2} V^{2}+h&=h_{0}
\end{align*}

\subsection{温度、压力与密度关系}

\begin{gather*}
    \frac{T_{0}}{T}=1+\frac{\gamma-1}{2} M^{2}\\ 
    \frac{p_{0}}{p}=\left(1+\frac{\gamma-1}{2} M^{2}\right)^{\frac{\gamma}{\gamma-1}}\\ 
    \frac{\rho_{0}}{\rho}=\left(1+\frac{\gamma-1}{2} M^{2}\right)^{\frac{1}{\gamma-1}}
\end{gather*}

\subsection{二维喷管中的流体}

考虑连续性方程:

\begin{gather*}
    \rho A V=(\rho+d \rho)(A+d A)(V+d V)\\ 
    0=\frac{d \rho}{\rho}+\frac{d A}{A}+\frac{d V}{V}
\end{gather*}

考虑动量守恒(牛顿第二定律):

\begin{equation*}
    \frac{d p}{\rho}+V d V=0
\end{equation*}

\subsection{喷管中的流体}

考虑

\begin{equation*}
    \frac{d p}{\rho}+V d V=0
\end{equation*}

代入

\begin{align*}
    a^2&=\frac{dp}{d\rho}\\ 
    M^2&=\frac{V^2}{a^2}
\end{align*}

可以得到

\begin{equation*}
    \frac{d \rho}{\rho}=-M^{2} \frac{d V}{V}
\end{equation*}

再把此式代回

\begin{equation*}
    \frac{d \rho}{\rho}+\frac{d A}{A}+\frac{d V}{V}=0
\end{equation*}

不难得到

\begin{equation*}
    \frac{d A}{A}=\left(M^{2}-1\right) \frac{d V}{V}
\end{equation*}

上式揭示了喷管截面积、马赫数与流体速度间的关系。$M=0$时,流体可以视为不可压缩流体;$0<M<1$时为亚音速流,渐缩喷管加速流体,渐扩喷管减速流体;$M=1$时为音速流,喷管截面积在此时取到最值;$M>1$时为超音速流,渐缩喷管减速流体,渐扩喷管加速流体。

为了得到超音速流体,需要使用缩放喷管(拉伐尔喷管)。

\subsection{喷管的截面积}

利用先前推导的温度、压力、密度关系,

\begin{align*}
    \frac{T_{0}}{T}&=1+\frac{\gamma-1}{2} M^{2}\\ 
    \frac{p_{0}}{p}&=\left(1+\frac{\gamma-1}{2} M^{2}\right)^{\frac{\gamma}{\gamma-1}}\\ 
    \frac{\rho_{0}}{\rho}&=\left(1+\frac{\gamma-1}{2} M^{2}\right)^{\frac{1}{\gamma-1}}
\end{align*}

在喉部位置,马赫数$M=1$,并且考虑空气的绝热指数$\gamma=1.4$,可以得到喉部的状态:

\begin{align*}
\frac{T^{*}}{T_{0}}&=0.833 \\ 
\frac{p^{*}}{p_{0}}&=0.528 \\ 
\frac{\rho^{*}}{\rho_{0}}&=0.634
\end{align*}

另一方面,利用连续性方程,统一代换为马赫数之比,即可得到任意截面面积与喉部截面积之比:

\begin{equation*}
    \frac{A}{A^{*}}=\left(\frac{1}{M}\right)\left(\frac{2}{(\gamma+1)}\left[1+\left(\frac{\gamma-1}{2}\right) M^{2}\right]\right)^{\frac{\gamma+1}{2(\gamma-1)}}
\end{equation*}

其中$A^*$为对应$M^*=1$的喉部截面积。

类似的,利用可压缩流体的伯努利方程(取参考点为滞止点),附加上绝热关系,可以得到,当地速度只与当地压力有关:

\begin{equation*}
    V=\sqrt{\frac{2 \gamma}{\gamma-1} \frac{p_{0}}{\rho_{0}}\left(1-\left(\frac{p}{p_{0}}\right)^{\frac{\gamma-1}{\gamma}}\right)}
\end{equation*}

