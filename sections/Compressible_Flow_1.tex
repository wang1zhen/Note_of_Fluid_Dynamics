\section{可压缩流体 - 基本公式}

对于不可压缩流体而言,$\rho=constant$;
可压缩流体则一般$\rho\neq constant$。

\subsection{可压缩流体的重要关系式}

\subsubsection{连续性方程(质量守恒)}

对于一个控制体,有

\begin{equation*}
    [\mbox{控制体内的质量改变率}]+[\mbox{通过控制体边界的质量}]=0
\end{equation*}

利用梯度算子,可以写作

\begin{equation*}
    \frac{\partial \rho}{\partial t}+\nabla(\rho \boldsymbol{V})=0
\end{equation*}

借助高斯散度定理,对等式两边积分,有

\begin{equation*}
    \frac{\partial}{\partial t} \iiint_{\Omega} \rho d \Omega+\oiint_{S} \rho \boldsymbol{V} \cdot d \boldsymbol{S}=0
\end{equation*}

\subsection{动量守恒}

考虑一个控制体的动量,

\begin{equation*}
    [\mbox{控制体动量的变化率}]=[\mbox{对控制体施加的外力}]
\end{equation*}

对控制体施加的合外力可以视为三个部分的和,压力、体积力和粘滞力的共同效果导致了控制体动量的改变。在三个方向的分量可以表示为:

\begin{align*}
    \frac{\partial(\rho u)}{\partial t}+\nabla (\rho u \boldsymbol{V})&=-\frac{\partial p}{\partial x}+\rho f_{x}+\left(F_{x}\right)_{viscous}\\
    \frac{\partial(\rho v)}{\partial t}+\nabla (\rho v \boldsymbol{V})&=-\frac{\partial p}{\partial y}+\rho f_{y}+\left(F_{y}\right)_{viscous}\\
    \frac{\partial(\rho w)}{\partial t}+\nabla (\rho w \boldsymbol{V})&=-\frac{\partial p}{\partial z}+\rho f_{z}+\left(F_{z}\right)_{visous}
\end{align*}

再次对等式两边积分,并利用高斯散度定理可得

\begin{equation*}
    \frac{\partial}{\partial t} \iiint_{\Omega} \rho \boldsymbol{V} d \Omega+\oiint_{S}(\rho \boldsymbol{V} \cdot \boldsymbol{d} \boldsymbol{S}) \boldsymbol{V}=-\oiint_{S} \rho d \boldsymbol{S}+\iiint_{\Omega} \rho f d \Omega+\boldsymbol{F}_{viscous}
\end{equation*}

左侧第一项可以视为控制体内动量的变化量,第二项则是通过控制体边界净增加/减少的动量;等式右侧第一项是作用在控制体上的压力之和(负号由于压力与单位面积向量的方向相反),第二项是控制体体积力之和,第三项则是总的粘滞力。

\subsubsection{能量守恒}

从热力学第一定律,我们有

\begin{equation*}
    \Delta E=Q+W
\end{equation*}

即

\begin{equation*}
    [\mbox{控制体内的能量变化率}]=[\mbox{传递给控制提的热功率}]+[\mbox{外界对控制体做功的功率}]
\end{equation*}

微分形式可以写为

\begin{equation*}
    \frac{\partial}{\partial t}\left[\rho\left(e+\frac{V^{2}}{2}\right)\right]+\nabla \cdot\left[\rho\left(e+\frac{V^{2}}{2}\right) \boldsymbol{V}\right]=\rho \dot{q}+\dot{\boldsymbol{Q}}_{\text {viscous}}^{\prime}-\nabla \cdot(p \boldsymbol{V})+\rho(\boldsymbol{f}, \boldsymbol{V})+\boldsymbol{W}_{\text {viscous}}^{\prime}
\end{equation*}

积分形式:

\begin{equation*}
    \frac{\partial}{\partial t} \iiint_{\Omega}\left[\rho\left(e+\frac{V^{2}}{2}\right)\right] d \Omega+\oiint_{S} \rho \rho\left(e+\frac{V^{2}}{2}\right) \boldsymbol{V} \cdot d S=\iiint_{\Omega} \rho \dot{q} d \Omega-\oiint_{S}(p \boldsymbol{V}) \cdot d \boldsymbol{S}+\iiint_{\Omega} \rho(f\cdot V) d \Omega+\dot{\boldsymbol{Q}}_{viscous}+\dot{\boldsymbol{W}}_{viscous}
\end{equation*}

\subsubsection{其他可用关系式}

理想气体状态方程

\begin{equation*}
    p=\rho RT
\end{equation*}

其中$R$为气体常数,

\begin{equation*}
    R=\frac{R_u}{MW}
\end{equation*}

内能关系:

\begin{equation*}
    e=c_vT
\end{equation*}

焓关系:

\begin{equation*}
    h=c_pT
\end{equation*}

绝热指数与热容关系:

\begin{gather*}
    \gamma=\frac{c_p}{c_v}\\
    c_p-c_v=R
\end{gather*}

绝热(等熵)过程:

\begin{equation*}
    \frac{p}{\rho^\gamma}=\mbox{const}
\end{equation*}


\subsection{声速}

\begin{equation*}
    a^{2}=\frac{\gamma p}{\rho}=\gamma R T
\end{equation*}

由此可定义马赫数

\begin{equation*}
    M=\frac{V}{a}=\mbox{Mach number}
\end{equation*}

类似地,可以定义水中的波速

\begin{equation*}
    c=\sqrt{gy}
\end{equation*}

式中,$g$为重力加速度,$y$为水深。同样也可定义Froude数,

\begin{equation*}
    \mbox { Froude number }=F=\frac{V}{\sqrt{g y}}
\end{equation*}